\chapter{Conclusions and future works}
\label{cap7}

In this final Chapter we recap the structure of the whole document and the development phases of the Pluto programming framework.
We also show the limits of Pluto and the possible future works thanks to which these limits could be overcome.

\section{Conclusions}

We have developed the Pluto programming framework, a system which allows to build nano-drones applications for indoor contexts, simply by graphically connecting blocks.

In this document we have fully described the development process of Pluto and the context surrounding our work:
\\

In Chapter \ref{cap1} we have given the general context and the general goals of the work together with a brief description of the Pluto development process.
\\

In Chapter~\ref{cap2} we have described the three main existing approaches for drone programming, also proposing existing examples for each one of them.
We have shown that no one of these approaches is suitable for our requirements, since we needed the concepts of \textit{Mission} and \textit{Trip}.
A Mission is a list of sensing tasks to be performed sequentially and a Trip is a movement from a point A to a point B in the environment to perform an Action.
We have also described the dataflow programming method, providing two existing examples of it.
Also in this case, we have shown that we needed a different approach, since we needed to use only a group of basic features for our work, while the existing solutions were too general and contained a lot of complex components.
\\

Chapter~\ref{cap3} is focused on the problems stemming from the indoor context and on the requirements deriving from it.
Starting from a motivating example application, in order to better explain the requirements and problems deriving from our work, we have shown the implementation problems deriving from using a Team-level approach for our system, also proposing the solutions to fix them.
Finally we have shown the technological limitations affecting our system, such as the indoor localization and nano-drone batteries problems.
\\

In Chapter~\ref{cap4} we have presented our solution for the research problems described in Chapter ~\ref{cap3}, the Pluto programming framework.
We have presented our programming model, that is to say the main entities and the relationships between them.
We have described the functionality of the blocks of the Pluto Graphical Editor, that are the basic elements that the programmer can connect to graphically build an application.
We also have described in details the two components of the Pluto framework:
the Graphical Editor, that is used by the programmer to graphically build an application and the Main Application, that is used by the final user to specify the sensing tasks to be performed.
We have describe the navigation system, that is the conjunction point between the Main Application and the drones team.
We finally have shown all the steps performed to arrive to the final system, showing all the previously implemented solutions which, once refined, brought us to the development of the Pluto programming framework.
\\

In Chapter~\ref{cap5} we have shown how the designed choices have been implemented technically, describing all the software and tools we used for the development of Pluto programming framework.
We have described:
The GEF framework, which we have used to implement the Pluto Graphical Editor.
The code generation process that creates a Java application from the graph built with the Pluto Graphical Editor.
The Object-Oriented programming model of the Pluto framework.
The runtime features of Pluto: the parallel architecture and the management of all the needed threads.
The SWING tool, which we have used to develop the Pluto Main Application.
The Crazyflie nano-quadcopter, which we have used to perform the sensing tasks of our prototype applications.
\\

In Chapter~\ref{cap6} we have described four already existing applications and three case study, and we have discussed on whether they can be developed or not with Pluto. 
We have proposed two exercises to real testers, in order to test "on the field" the effective usability of Pluto:
the first one deals with the Graphical Editor, the second one with the Main Application.
Then we have proposed a third exercise, in which we ask the users to directly use the API of the Crazyflie nano-quadcopter, shown in Section \ref{crazyflie}, to make it move from a point A to a point B.
We have also proposed a survey to the users, in order to have opinions on the framework and possibly to improve it with the suggestions of the testers.
We also have measured the software and hardware consumption metrics required by Pluto, in order to evaluate the effective impact of Pluto on an ordinary computing machine.

\section{Pluto limits and future works}

In this Section we show the limits of the Pluto programming framework.

There are two type of limits:
the limits in the implementation can possibly be overcome by modifying the source code and/or adding new features, or changing the whole model of the system.
The technological limits cannot be overcome in the present, and only research and studies can find a way to improve or find new technologies which would solve these problems.


The PURSUE application, described in Section \ref{PURSUE}, put in evidence the Pluto main limitation: the immediate execution of actions in response to instantaneous events.

As already explained in chapter \ref{cap4}, the Pluto system allows the drones to perform their actions only at the end of the Trip, that is a movement from a point A to a point B in the environment.
For example, if the Drone has to take a picture in a specific location, it flies from the ground station to that location and then it takes the picture.

There is no way to actively perform actions reacting on events:
so, as already explained, this is the problem of the PURSUE application, which requires to actively follow a moving object when it enters in the camera range.

This is an hint for the future expansion of Pluto, in order to manage also this kind of applications.
\\

As already explained in chapter \ref{cap3}, the actuation in indoor contexts is tricky because of the localization problem.
We showed some IPS methods, but still they are not as efficient and standardized as GPS.
They introduce latency in the localization mechanism and their precision is lowered by physical obstacles, roofs and ceilings.
\\

In this direction, research and future studies will certainly find a better indoor localization method, and Pluto will take advantage from it, since it has an architecture that is decoupled from the particular localization method.
So, when this new method will be implemented, it will be easily integrated with Pluto.
This is an important missing feature, since the actuation tasks performed by the drones completely relies and depends on a localization base:
in order to send a drone in a specific location to take a picture there is need of a method that precisely indicates that location.
\\

Research and future studies will also find a way to improve the capacity of the nano-drones batteries. 
Nowadays their duration is approximately of 7 minutes, with a recharge time of 20 minutes.
This is a great limitation, because the programmer is forced to develop applications where the sensing tasks must be performed within this limited amount of time.
\\

Finding a solution to the limitation of the instantaneous actuation, together with new technological discoveries that will improve the drones battery duration and find a good and stable indoor localization method, can greatly enrich the Pluto programming framework.
\\

