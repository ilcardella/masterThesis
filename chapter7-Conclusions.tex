\chapter{Conclusions and future works}
\label{cap7}

%Si mostrano le prospettive future di ricerca nell'area dove si \`e svolto il lavoro. Talvolta questa sezione pu\`o essere l'ultima sottosezione della precedente. Nelle conclusioni si deve richiamare l'area, lo scopo della tesi, cosa \`e stato fatto,come si valuta quello che si \`e fatto e si enfatizzano le prospettive future per mostrare come andare avanti nell'area di studio.

\section{Conclusions}

We have developed the Pluto Programming Framework, a system which allows to build indoor applications by simply graphically connecting blocks.

Each one of these blocks, described in section \ref{dataFlow}, contains the implementation of a specific functionality.

So the programmer can build an application just deciding which features he needs and consequently connecting the right blocks.

The final user decides the sensing tasks that have to be performed by the drones thanks to the user interface.

The system takes care of assigning the drones to each sensing task, and this really simplify the work for the final user, which has only to decide a list of sensing tasks without dealing with the drones dispatching. 

The main innovations with respect to the actual state of the art, fully described in chapter \ref{cap2}, are represented by the indoor context and the graphical editor.

Indeed, there exist systems similar to Pluto, but they all manage outdoor applications, where the GPS can be used for localization and big drones for actuation of the sensing tasks.

Furthermore, the building of applications through connection of blocks really simplifies the development process, still allowing the programmer to insert custom code.

In chapter \ref{cap6} we fully evaluated Pluto, confronting it with other similar systems, trying to apply it to existing applications, proposing its use to real people and asking them for feedback through a survey, and finally measuring its performance, through bot hardware and software metrics.

We state that Pluto is a useful programming abstraction, which allows the development of a great variety of applications in a simple and fast way.
As any other system, Pluto has limits, both for its implementation and for technological issues, which we show in the next section.

\newpage

\section{Pluto limits and future works}

The PURSUE application, described in section \ref{PURSUE}, put in evidence the Pluto main limitation: the immediate execution of actions in response to instantaneous events .
\\

As already explained in chapter \ref{cap4}, the Pluto system allows the drones to perform their actions only at the end of the trip, that is a movement from a point A to a pint B in the environment.
For example, if the Drone has to take a picture in a specific location, it will fly from the ground station to that location and then take the picture.
\\

There is no way to actively performing some action, reacting on events:
so, as already explained, this is the problem of the PURSUE application, which requires to actively follow a moving object when it enters in the camera range.
\\

This is an hint for the future expansion of Pluto, in order to manage also this kind of applications.

As already explained in chapter \ref{cap3}, the actuation in indoor contexts is tricky because of the localization problem.
We showed some IPS methods, but still they are not as efficient and standardized as GPS.
They introduce latency in the localization mechanism and their precision is lowered by physical obstacles, roofs and ceilings.
\\

In this direction, research and future studies will certainly find a better indoor localization method, and Pluto will take advantages from it.
Indeed the actuation tasks performed by the drones completely relies and depends on a localization base: indeed, in order to send a drone in a specific location to take a picture we need a method to precisely indicate that location.
\\

Research and future studies will also find a way to improve the capacity of the nano-drones batteries. 
Nowadays their duration is approximately of 7 minutes, with a recharge time of 20 minutes.
This is a great limitation, because the programmer is forced to develop applications where the sensing tasks must be performed within this limited amount of time.
\\

Finding a solution to the limitation of the instantaneous actuation, together with new technological discoveries that will improve the drones battery duration and find a good and stable indoor localization method can greatly enrich the Pluto programming framework.
\\

With these future improvement the Pluto programming framework will be able to manage almost every kind of drones application.







