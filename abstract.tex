\chapter*{Abstract}

\addcontentsline{toc}{chapter}{Abstract}

Drone-teams programming is rapidly expanding since it allows to automatically perform a lot of useful tasks.
Existing systems are able to manage a group of drones and dispatching them in the environment. 
All these systems deal with outdoor applications, where medium/big sized drones collaboratively perform tasks making use of the Global Positioning System (GPS) to navigate in the space.
We want to deal with the indoor context, and no one of the existing system is fully suitable for it.
Indeed, the indoor context implies applications with different requirements compared to the outdoor ones:
there is need for a small number of drones(5/10), each one performing a different action independently from the others, while in the outdoor environment generally there is need for a large number of drones to perform the same action.
To address this problem, we propose the concept of \textit{Trip}, that is nothing but a movement of a drone from a point A to a point B at the end of which an action (picture,measurement etc.) is performed. 
No one of the existing programming systems provides the concept of Trip.
Furthermore the second important goal is to make the system autonomous in the choice of the drone to allocate for each Trip. 
This means that the user does not need to take this important decision.
We propose the Pluto programming framework as a solution to these problems. 
It consists of two main components: the Graphical Editor and the Main Application.
With the former a programmer can build an application by simply connecting functional blocks.
Each block implements a precise functionality, for example there is one that chooses the drones to assign to each sensing task, one that manages the priority of the sensing tasks etc.
Then, through the "Generate code" command, the Pluto Graphical Editor generates the source code of the second main component, the Pluto Main Application.
The final user uses this generated Pluto Main Application to define and execute the sensing tasks.
The key strength of our programming framework is its scalable architecture, in which the central brain is independent from the particular navigation API, which means that the system manages the dispatching of drones and their failures independently of the specific navigation algorithm.
We evaluated the Pluto programming framework by proposing its use to real testers and asking them for a feedback. 
Moreover we measured its software and hardware performance and also tried to implement some existing applications with it.
After the evaluation we noticed that, even if it has some limits, Pluto is useful to simplify the developing of drone-teams applications.