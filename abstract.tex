\chapter*{Abstract}

\addcontentsline{toc}{chapter}{Abstract}

Autonomous drones are performing a revolution in the field of mobile sensing:
various type of drones are used to perform a great number of applications, since they can carry rich sensor payloads, such as cameras and microphones.
Often there is a simple abstraction which allows drones navigation: they can be controlled through smartphones and tablets interfaces or by setting waypoints.
Drones can greatly extend the capabilities of traditional sensing systems while simultaneously reducing cost.
They can monitor a farmer’s crops, manage parking spaces, or monitor underwater telecommunication systems more practically and/or more cheaply than stationary sensors.
The great innovation brought by drones is that they offer direct control on where to sample the environment; this was impossible with previous mobile sensing systems that could only passively sample the environment, laying on the mobility of smartphones or vehicles.
So, the drones can greatly extend the field of mobile sensing, allowing the programmers to create a great number of applications, unthinkable and impossible to develop with the already existing technologies.
