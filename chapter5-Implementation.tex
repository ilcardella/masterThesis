\chapter{Implementation}
\label{cap5}



\section{Graphical editor}\label{editor}

Description of the GEF Framework with code examples
\\
Graphical Editing Framework (GEF) provides a powerful foundation for creating editors for visual editing of arbitrary models. Its effectiveness lies in a modular build, fitting use of design patterns, and decoupling of components that comprise a full, working editor. To a newcomer, the sheer number and variety of concepts and techniques present in GEF may feel intimidating. However, once learned and correctly used, they help to develop highly scalable and easy to maintain software. This section aims to provide a gentle yet comprehensive introduction to GEF. It describes our Pluto Graphical Editor.


\section{Code generation}\label{codeGeneration}

Description of the generation process of the code from diagram

\section{Object-oriented approach}\label{oomodel}

Why we decided to use JAVA, How we implement entities, UML diagrams of classes, MVC pattern

\section{Runtime Management}\label{runtimeMng}

Description of the management of threads, pub/sub pattern and log procedures

\section{User interface}\label{interface}

Swing library is an official Java GUI toolkit released by Sun Microsystems. It is used to create Graphical user interfaces with Java.
The main characteristics of the Swing toolkit:
\begin{itemize}
\item platform independent
\item customizable
\item extensible
\item configurable
\item lightweight
\end{itemize}

Swing is an advanced GUI toolkit. It has a rich set of widgets. From basic widgets like buttons, labels, scrollbars to advanced widgets like trees and tables. Swing itself is written in Java.
Swing is a part of JFC, Java Foundation Classes. It is a collection of packages for creating full featured desktop applications.

There are basically two types of widget toolkits: \textit{Lightweight} and \textit{Heavyweight}.
A heavyweight toolkit uses OS's API to draw the widgets. For example Borland's VCL is a heavyweight toolkit since it depends on WIN32 API, the built in Windows application programming interface.
As already said, Swing is a lightweight toolkit since it paints its own widgets.

We used Swing to develop the user interface of the Pluto programming framework.
As already shown in section \ref{plutoMainApp}, the interface is composed by three main sections: the Missions Page, the Trips Page and the Monitor Page.
In the proceeding of this section, we show the code of the three pages, commenting the main Swing features used to build them.
\newpage

\textbf{Missions Page}
\\

Here we describe the main Swing features of the Missions Page, whose graphical result is shown in section \ref{plutoMainApp}:

\begin{lstlisting}
	private JList<String> list;
	private JButton remallbtn;
	private JButton addbtn;
	private JButton renbtn;
	private JButton delbtn;
	private JButton tpsbtn;
	private JButton okbtn;

	public MissionsPage() {
		initUI();
	}

	private void createList() {

		DefaultListModel<String> model = new DefaultListModel<String>();
		list = new JList<String>(model);
		...
	}

	private void createButtons() {

		remallbtn = new JButton("Remove All");
		addbtn = new JButton("Add Mission");
		renbtn = new JButton("Rename");
		delbtn = new JButton("Delete");
		tpsbtn = new JButton("Set Trips");
		okbtn = new JButton("Monitor Page");

	}
    
\end{lstlisting}

This is the basic structure of the Missions Page:
the \textit{JList} component is the list of the Missions created by the user. At the beginning it's empty and it'll be filled with the string representing the name of the Missions the user choose to create.
It is created thanks to the \textit{createList()} method.
There is a number of \textit{JButton} components:
these are all the buttons that the user can click on the Missions Page, and each of them perform a different action, as already explained in section \ref{plutoMainApp}.
They are created thanks to the \textit{createButtons()} method.
The constructor of the Missions page only calls the \textit{initUI()} method, whose implementation is shown in the following code snippet:

\begin{lstlisting}
	private void initUI() {

		createList();
		createButtons();
		Container pane = getContentPane();
		GroupLayout gl = new GroupLayout(pane);
		pane.setLayout(gl);


		gl.setHorizontalGroup(
        			...
				)
                
        );

		gl.setVerticalGroup(
					...
				)

		);

		...

		setTitle("Pluto-Missions Page");
		setSize(1000, 800);
        ...
		setDefaultCloseOperation(EXITONCLOSE);
	}
\end{lstlisting}

First of all the \textit{createList()} and \textit{createButtons()} methods are called, in order to create the empty list of Missions and all the buttons of the Missions Page.
We chose a \textit{GroupLayout} for the visualization of the Components in the page, but there exists a lot of Swing layouts that can be used.
Through the \textit{setHorizontalGroup} and \textit{setVerticalGroup} methods, we simply chose the graphical disposition of the components in the Missions Page.
Finally, we set the title and the size of the Missions Page window, and, through the \textit{setDefaultCloseOperation(EXITONCLOSE)} method, we make the Missions page close when one clicks on the close icon on the top of the window.

The rest of the code of the Missions page deals with the reaction of the model when an action on a component is performed.
For example, when the \textit{Add Mission} button is clicked, a little window asking the name for the Mission appears, followed by another window asking if the Mission must be repeated.
Since in this section we want to show only the graphical part of the Pluto Main Application, and this code involves the whole MVC pattern, we decide to not show it, because it would be too specific and we would have to show a lot of different features of the system, involving also the Model and Controller parts.
\\

\textbf{Trips Page}
\\

In this section we show the main Swing features of the Trips Page:

\begin{lstlisting}
	...
    
	private JList<String> list;
	private JList<String> tripList;

	...

	private ImageIcon icon;
	private JLabel label;
    
	private JButton ok;
	private JButton delete;
	private JButton deleteOne;
	private JTextField text;
    
	...

	public TripsPage(...) {
		this.nameMission = name;
		this.tripsMap = map;

		try {
			initUI();
		} catch (IOException e) {
			e.printStackTrace();
		}
	}
\end{lstlisting}

This is the basic structure of the Trips page, which is used to create the Trips list of each Mission created with the Missions Page.
We have two \textit{Jlist} components: \textit{list} is the list of actions that the user can drag and drop on the map, the one displayed on the top of the page, and \textit{tripList} is the list of the Trips created by the user, initially empty, displayed on the south-east corner of the page.
The \textit{ImageIcon} component is later filled with the map on which the user can drag and drop the action to build the various tips.
The \textit{JLabel} component in just a container of the map element.
The three \textit{Jbutton} components are the buttons displayed in the Trips Page.
The constructor only calls the \textit{initUI()} method, as for each page.

\begin{lstlisting}
	private void createActionList() {
		actionListModel = new DefaultListModel<String>();
		list = new JList<String>(actionListModel);
        
        ...

		list.setDragEnabled(true);
		list.setTransferHandler(new TransferHandler("text"));
		ds = new DragSource();
		ds.createDefaultDragGestureRecognizer(list, DnDConstants.ACTION_COPY,
				this);

		// add the action names
		for (Action a : Action.values())
			actionListModel.addElement(a.toString());
	}

	private void createTripList() {
		tripListModel = new DefaultListModel<String>();
		tripList = new JList<String>(tripListModel);
		...
        }
        
\end{lstlisting}

The \textit{createActionList()} method takes care of correctly fill and display the list of actions the user can drag and drop on the map.
The lines from 7 to 11 set the list of actions \textit{list} as a drag source component.
Lines 14 and 15 fill the action list with the correct values: Take photo, Measure, Pick Item and Release item.
The \textit{createTripList()} method takes care of creating an empty list that will be filled with the trips created by the user.

\begin{lstlisting}
	public final void initUI() throws IOException {

		setLayout(new BorderLayout());

		icon = new ImageIcon("Map/casa.gif");
		label = new JLabel(icon);

		createActionList();
		createTripList();
        
        ...

		JScrollPane actionsPane = new JScrollPane(list);
		JScrollPane tripsPane = new JScrollPane(tripList);
		JPanel imagePane = new JPanel();
		JPanel buttonsPane = new JPanel();

		ok = new JButton("Ok");
		delete = new JButton("Delete all");
		deleteOne = new JButton("Delete trip");

		imagePane.add(label);
		imagePane.setTransferHandler(new TransferHandler("text"));
		new MyDropTargetListener(imagePane);
        
		...

		buttonsPane.add(ok);
		buttonsPane.add(delete);
		buttonsPane.add(deleteOne);

		getContentPane().add(imagePane, BorderLayout.WEST);
		getContentPane().add(actionsPane, BorderLayout.NORTH);
		getContentPane().add(buttonsPane, BorderLayout.SOUTH);
		getContentPane().add(tripsPane, BorderLayout.EAST);

		setTitle("Pluto-Trips Page (" + nameMission + ")");
		setSize(700, 600);
        ...
		setVisible(true);

	}
    
\end{lstlisting}

In line 5 we assign the picture of the map to the previously defined \textit{icon} element.
In line 6 this element is put on the \textit{JLabel} container.
Lines 8 and 9 create the trips and actions lists.
Lines from 13 to 16 simply creates the containers for all the components of the Trips Page.
Lines from 18 to 20 create the buttons.
Lines from 22 to 24 sets the map element as the drop target of the dragged actions.
Lines from 28 to 30 simply add the buttons to their container.
Lines from 32 to 35 displays each container of the components in a different part of the Trips Page.
Line 37 assigns a title to each Trips page, composed by the string "Pluto-Trips Page" and the name of the Mission of which the user is setting the trips.
Line 38 simply sets the size of the Trips Page.
Finally, line 40 simply makes the Trips Page appear on the screen.
\\

Once again, the remaining part of the code of the Trips Page involves all the reaction mechanisms of the components, such as all the methods needed to manage the correct functioning of the drag and drop mechanism.
Since we are not interested in showing this details in this section, we simply skip them.

\newpage

\textbf{Monitor Page}
\\

Here the main Swing features of the Monitor Page are shown:

\begin{lstlisting}
	private JButton start;
	private JButton stop;
	private JButton back;
	private JTable table;
	private JTextArea text;
    private jTable table;

	public MonitorPage() {

		initUI();
		
	}
\end{lstlisting}

Here all the components of the Monitor page are declared:
we have a \textit{JTable} where will be displayed information on all the trips executing, a \textit{JTextArea} where the log of the execution will be displayed and three \textit{Jbutton} components.


\begin{lstlisting}
public final void initUI() {

	...

	JScrollPane tablePane = new JScrollPane(table);
	DefaultTableModel model = new DefaultTableModel();
    table = new JTable(model);
	model.addColumn("Mission");
    model.addColumn("Mission status");
	model.addColumn("Drone");
	model.addColumn("Drone status");
	model.addColumn("Trip");
	model.addColumn("Trip Status");

	...
    
    JScrollPane textPane = new JScrollPane(text);
	text = new JTextArea();
	text.setBackground(Color.LIGHT_GRAY);
	text.setForeground(Color.RED);

	...

	JPanel buttonsPane = new JPanel();
	start = new JButton("Start");
	stop = new JButton("Stop");
	back = new JButton("Back"); 

	buttonsPane.add(start);
	buttonsPane.add(stop);
	buttonsPane.add(back);

	getContentPane().add(tablePane, BorderLayout.NORTH);
	getContentPane().add(buttonsPane, BorderLayout.EAST);
	getContentPane().add(textPane);

	setTitle("Pluto - Monitor Page");
	setSize(1000, 800);
			
	}
\end{lstlisting}


In lines 5, 17 and 24 the containers of the \textit{table}, log console \textit{text} and buttons, are defined respectively.
Lines from 6 to 13 define the table and all his column. Each row will display the information of a Mission, such as the trip currently executing and the drone assigned to that trip.
Lines 18 to 20 define the console where the log of the execution is showed.
Lines 24 to 31 creates and add the buttons to their container.
Lines 33 to 35 build the Monitor Page, adding each container in a different part of the page.
Finally, lines 37 and 38 sets the title and the size of the page.
\\

Also for the Monitor Page, the rest of the code deals with mechanisms which are not interesting for this section, so we don't show it here.


\section{The crazyflie nano-quadcopter}\label{crazyflie}

Description of crazyflie API 
