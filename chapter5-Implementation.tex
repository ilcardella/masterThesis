\chapter{Implementation}
\label{cap5}



\section{Graphical editor}\label{editor}

Description of the GEF Framework with code examples
\\
Graphical Editing Framework (GEF) provides a powerful foundation for creating editors for visual editing of arbitrary models. Its effectiveness lies in a modular build, fitting use of design patterns, and decoupling of components that comprise a full, working editor. To a newcomer, the sheer number and variety of concepts and techniques present in GEF may feel intimidating. However, once learned and correctly used, they help to develop highly scalable and easy to maintain software. This section aims to provide a gentle yet comprehensive introduction to GEF. It describes our Pluto Graphical Editor.


\section{Code generation}\label{codeGeneration}

Description of the generation process of the code from diagram

\section{Object-oriented approach}\label{oomodel}

Why we decided to use JAVA, How we implement entities, UML diagrams of classes, MVC pattern

\section{Runtime Management}\label{runtimeMng}

Description of the management of threads, pub/sub pattern and log procedures

\section{User interface}\label{interface}

We chose the Swing tool to develop the Pluto User Interface, already described in section \ref{plutoMainApp}.
We made this choice since Swing is well known to us.
Indeed we used it for the development of many academic projects, where we noticed that it allows to build graphical interface in a very fast and easy way and to add a great variety of components.
\\

Swing library is an official Java GUI toolkit released by Sun Microsystems. It is used to create Graphical user interfaces with Java.
The main characteristics of the Swing toolkit:
\begin{itemize}
\item platform independent
\item customizable
\item extensible
\item configurable
\item lightweight
\end{itemize}

Swing is an advanced GUI toolkit. It has a rich set of widgets. From basic widgets like buttons, labels, scrollbars to advanced widgets like trees and tables. Swing itself is written in Java.
Swing is a part of JFC, Java Foundation Classes. It is a collection of packages for creating full featured desktop applications.

There are basically two types of widget toolkits: \textit{Lightweight} and \textit{Heavyweight}.
A heavyweight toolkit uses OS's API to draw the widgets. For example Borland's VCL is a heavyweight toolkit since it depends on WIN32 API, the built in Windows application programming interface.
As already said, Swing is a lightweight toolkit since it paints its own widgets.

\section{The crazyflie nano-quadcopter}\label{crazyflie}

Description of crazyflie API 
