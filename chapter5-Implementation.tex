\chapter{Implementation}
\label{cap5}



\section{Graphical editor}\label{editor}

Description of the GEF Framework with code examples
\\
Graphical Editing Framework (GEF) provides a powerful foundation for creating editors for visual editing of arbitrary models. Its effectiveness lies in a modular build, fitting use of design patterns, and decoupling of components that comprise a full, working editor. To a newcomer, the sheer number and variety of concepts and techniques present in GEF may feel intimidating. However, once learned and correctly used, they help to develop highly scalable and easy to maintain software. This section aims to provide a gentle yet comprehensive introduction to GEF. It describes our Pluto Graphical Editor.


\section{Code generation}\label{codeGeneration}

Description of the generation process of the code from diagram

\section{Object-oriented approach}\label{oomodel}

Why we decided to use JAVA, How we implement entities, UML diagrams of classes, MVC pattern

\section{Runtime Management}\label{runtimeMng}

Description of the management of threads, pub/sub pattern and log procedures

\section{User interface}\label{interface}

Swing library is an official Java GUI toolkit released by Sun Microsystems. It is used to create Graphical user interfaces with Java.
The main characteristics of the Swing toolkit:
\begin{itemize}
\item platform independent
\item customizable
\item extensible
\item configurable
\item lightweight
\end{itemize}

Swing is an advanced GUI toolkit. It has a rich set of widgets. From basic widgets like buttons, labels, scrollbars to advanced widgets like trees and tables. Swing itself is written in Java.
Swing is a part of JFC, Java Foundation Classes. It is a collection of packages for creating full featured desktop applications.

There are basically two types of widget toolkits: \textit{Lightweight} and \textit{Heavyweight}.
A heavyweight toolkit uses OS's API to draw the widgets. For example Borland's VCL is a heavyweight toolkit since it depends on WIN32 API, the built in Windows application programming interface.
As already said, Swing is a lightweight toolkit since it paints its own widgets.

We used Swing to develop the user interface of the Pluto programming framework.
As already shown in section \ref{plutoMainApp}, the interface is composed by three main sections: the Missions Page, the Trips Page and the Monitor Page.
In the proceeding of this section, we show the code of the three pages, commenting the main Swing features used to build them.
\newpage

\textbf{Missions Page}
\\

Here we describe the main Swing features of the Missions Page, whose graphical result is shown in section \ref{plutoMainApp}:

\begin{lstlisting}

	private JList<String> list;
	private JButton remallbtn;
	private JButton addbtn;
	private JButton renbtn;
	private JButton delbtn;
	private JButton tpsbtn;
	private JButton okbtn;

	public MissionsPage() {
		initUI();
	}

	private void createList() {

		DefaultListModel<String> model = new DefaultListModel<String>();
		list = new JList<String>(model);
		...
	}

	private void createButtons() {

		remallbtn = new JButton("Remove All");
		addbtn = new JButton("Add Mission");
		renbtn = new JButton("Rename");
		delbtn = new JButton("Delete");
		tpsbtn = new JButton("Set Trips");
		okbtn = new JButton("Monitor Page");

	}

\end{lstlisting}

This is the basic structure of the Missions Page:
the \textit{JList} component is the list of the Missions created by the user. At the beginning it's empty and it'll be filled with the string representing the name of the Missions the user choose to create.
It is created thanks to the \textit{createList()} method.
There is a number of \textit{JButton} components:
these are all the buttons that the user can click on the Missions Page, and each of them perform a different action, as already explained in section \ref{plutoMainApp}.
They are created thanks to the \textit{createButtons()} method.
The constructor of the Missions page only calls the \textit{initUI()} method, whose implementation is shown in the following code snippet:

\begin{lstlisting}

	private void initUI() {

		createList();
		createButtons();
		Container pane = getContentPane();
		GroupLayout gl = new GroupLayout(pane);
		pane.setLayout(gl);


		gl.setHorizontalGroup(
        			...
				)
                
        );

		gl.setVerticalGroup(
					...
				)

		);

		...

		setTitle("Pluto-Missions Page");
		setSize(1000, 800);
        ...
		setDefaultCloseOperation(EXITONCLOSE);
	}
\end{lstlisting}
First of all the \textit{createList()} and \textit{createButtons()} methods are called, in order to create the empty list of Missions and all the buttons of the Missions Page.
We chose a \textit{GroupLayout} for the visualization of the Components in the page, but there exists a lot of Swing layouts that can be used.
Through the \textit{setHorizontalGroup} and \textit{setVerticalGroup} methods, we simply chose the graphical disposition of the components in the Missions Page.
Finally, we set the title and the size of the Missions Page window, and, through the \textit{setDefaultCloseOperation(EXITONCLOSE)} method, we make the Missions page close when one clicks on the close icon on the top of the window.

The rest of the code of the Missions page deals with the reaction of the model when an action on a component is performed.
For example, when the \textit{Add Mission} button is clicked, a little window asking the name for the Mission appears, followed by another window asking if the Mission must be repeated.
Since in this section we want to show only the graphical part of the Pluto Main Application, and this code involves the whole MVC pattern, we decide to not show it, because it would be too specific and we would have to show a lot of different features of the system, involving also the Model and Controller parts.
\\

\textbf{Trips Page}
\\

jhsd

\section{The crazyflie nano-quadcopter}\label{crazyflie}

Description of crazyflie API 
