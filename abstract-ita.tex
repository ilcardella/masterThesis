\chapter*{Sommario}

\addcontentsline{toc}{chapter}{Sommario}

La programmazione di team di droni è in rapida espansione, in quanto permette di eseguire in maniera automatica un gran numero di azioni utili.
I sistemi esistenti sono in grado di gestire un gruppo di droni e farli navigare nell'ambiente.
Tutti questi sistemi trattano applicazioni outdoor, nelle quali droni di dimensione medio / grande collaborano insieme per svolgere varie azioni, utilizzandoil sistema di posizionamento globale (GPS) per navigare nello spazio.
Noi vogliamo affrontare il contesto indoor, e nessuno dei sistemi esistente è completamente adatto per esso.
Infatti, il contesto indoor implica lo sviluppo di applicazioni con requisiti differenti rispetto a quello outdoor:
c'è bisogno di un piccolo numero di droni (5/10), ciascuno che esegua un'azione diversa, indipendentemente da tutti gli altri, mentre in ambiente outdoor generalmente c'è bisogno di un gran numero di droni che eseguano la stessa azione.
Per risolvere questo problema proponiamo il concetto di \textit{Trip}, che non è altro che un movimento di un drone da un punto A ad un punto B al termine del quale viene eseguita un'azione (scatto di una foto, misurazione di una grandezza fisica ecc.).
Nessuno dei modelli di programmazione esistenti fornisce il concetto di Trip.
Inoltre, il secondo obiettivo importante è quello di rendere il sistema autonomo nella scelta del drone da assegnare ad ogni Trip.
Ciò significa che l'utente non deve prendere questa importante decisione.
Per rispondere a tutti questi problemi, abbiamo sviluppato il framework di programmazione Pluto.
Pluto ha due componenti principali: il Graphical Editor e la Main Application.
Con il primo, un programmatore può costruire un'applicazione semplicemente collegando dei blocchi funzionali.
Ogni blocco implementa una funzionalità precisa, per esempio ce n'è uno che sceglie i droni da assegnare a ciascun Trip, un altro che gestisce la priorità dei Trip ecc.
Quindi, grazie alla funzionalità di generazione del codice, il Pluto Graphical Editor genera il codice sorgente del secondo componente principale, la Pluto Main Application.
L'utente finale utilizza questa Main Application per definire ed eseguire i compiti nell'ambiente.
Il punto di forza del nostro framework è la sua architettura scalabile che lo rende indipendente dalle API di navigazione utilizzate.
Ciò significa che il sistema è capace di gestire l'invio dei droni e i loro fallimenti indipendentemente dall'algoritmo di navigazione specifico.
Abbiamo testato l'utilizzo del framework di programmazione Pluto, proponendone l'uso a dei tester reali e chiedendo loro un feedback.
Inoltre abbiamo misurato le prestazioni software e hardware e abbiamo cercato di implementare alcune applicazioni esistenti con esso.
Grazie alla fase di valutazione, abbiamo notato che, anche se è caratterizzato da alcuni limiti, Pluto è utile per semplificare lo sviluppo di applicazioni per team di droni.