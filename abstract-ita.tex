\chapter*{Sommario}

\addcontentsline{toc}{chapter}{Sommario}

I droni autonomi stanno compiendo una vera e propria rivoluzione nel campo del rilevamento mobile:
vari tipi di droni sono utilizzati per eseguire un gran numero di applicazioni, in quanto possono trasportare una gran quantità di dati raccolti grazie a sensori quali fotocamere e microfoni.
Spesso vi è una semplice astrazione che permette ai droni di navigare nell'ambiente: possono essere controllati tramite interfacce grafiche per smartphone e tablet o impostando dei waypoint.
I droni possono notevolmente estendere le capacità dei sistemi di rilevamento tradizionali riducendo al contempo i costi.
Si possono monitorare le colture di un agricoltore, gestire i parcheggi, o monitorare i sistemi di telecomunicazione sottomarini in maniera più pratica e/o più a buon mercato rispetto ai sensori fissi.
La grande innovazione portata dai droni è che offrono un controllo diretto sui punti da monitorare  nell'ambiente; questo era impossibile con i precedenti sistemi di rilevamento mobili che potevano solo passivamente rilevare l'ambiente, basandosi sulla mobilità di smartphone o veicoli.
Così, i droni possono estendere notevolmente il campo del rilevamento mobile, permettendo ai programmatori di creare un gran numero di applicazioni, impensabili precedentemente e impossibili da sviluppare con le tecnologie già esistenti.